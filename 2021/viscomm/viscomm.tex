% Options for packages loaded elsewhere
\PassOptionsToPackage{unicode}{hyperref}
\PassOptionsToPackage{hyphens}{url}
%
\documentclass[
]{article}
\usepackage{amsmath,amssymb}
\usepackage{lmodern}
\usepackage{ifxetex,ifluatex}
\ifnum 0\ifxetex 1\fi\ifluatex 1\fi=0 % if pdftex
  \usepackage[T1]{fontenc}
  \usepackage[utf8]{inputenc}
  \usepackage{textcomp} % provide euro and other symbols
\else % if luatex or xetex
  \usepackage{unicode-math}
  \defaultfontfeatures{Scale=MatchLowercase}
  \defaultfontfeatures[\rmfamily]{Ligatures=TeX,Scale=1}
\fi
% Use upquote if available, for straight quotes in verbatim environments
\IfFileExists{upquote.sty}{\usepackage{upquote}}{}
\IfFileExists{microtype.sty}{% use microtype if available
  \usepackage[]{microtype}
  \UseMicrotypeSet[protrusion]{basicmath} % disable protrusion for tt fonts
}{}
\makeatletter
\@ifundefined{KOMAClassName}{% if non-KOMA class
  \IfFileExists{parskip.sty}{%
    \usepackage{parskip}
  }{% else
    \setlength{\parindent}{0pt}
    \setlength{\parskip}{6pt plus 2pt minus 1pt}}
}{% if KOMA class
  \KOMAoptions{parskip=half}}
\makeatother
\usepackage{xcolor}
\IfFileExists{xurl.sty}{\usepackage{xurl}}{} % add URL line breaks if available
\IfFileExists{bookmark.sty}{\usepackage{bookmark}}{\usepackage{hyperref}}
\hypersetup{
  pdftitle={test},
  hidelinks,
  pdfcreator={LaTeX via pandoc}}
\urlstyle{same} % disable monospaced font for URLs
\usepackage[margin=1in]{geometry}
\usepackage{graphicx}
\makeatletter
\def\maxwidth{\ifdim\Gin@nat@width>\linewidth\linewidth\else\Gin@nat@width\fi}
\def\maxheight{\ifdim\Gin@nat@height>\textheight\textheight\else\Gin@nat@height\fi}
\makeatother
% Scale images if necessary, so that they will not overflow the page
% margins by default, and it is still possible to overwrite the defaults
% using explicit options in \includegraphics[width, height, ...]{}
\setkeys{Gin}{width=\maxwidth,height=\maxheight,keepaspectratio}
% Set default figure placement to htbp
\makeatletter
\def\fps@figure{htbp}
\makeatother
\setlength{\emergencystretch}{3em} % prevent overfull lines
\providecommand{\tightlist}{%
  \setlength{\itemsep}{0pt}\setlength{\parskip}{0pt}}
\setcounter{secnumdepth}{-\maxdimen} % remove section numbering
\ifluatex
  \usepackage{selnolig}  % disable illegal ligatures
\fi

\title{test}
\author{}
\date{\vspace{-2.5em}}

\begin{document}
\maketitle

\hypertarget{requirements}{%
\section{Requirements}\label{requirements}}

Topics:

\begin{itemize}
\item
  How should visualization adapt to its new, more diverse audience?
  Visualization for communication addresses an audience that is much
  more varied in demographics and literacy than visualization for
  analysis. When do visualizations communicate successfully, and how can
  we measure that success? Methods might include web analytics,
  behavioral studies, eye tracking, or even galvanic skin response.
\item
  How can practitioners build visualizations that communicate
  successfully? Are there models that can guide effective communicative
  visualization, possibly derived from theories of aesthetics, memory,
  metaphor, or persuasion?
\item
  Are there certain visualization techniques (like ``chart junk'') that
  are particularly helpful for communication? How well do they work in
  concert? What tools do practitioners need to help them build
  visualizations for communication? What are typical practitioner
  workflows, and which parts of them are most challenging?
\item
  Which application areas are still emerging for communicative
  visualization? How would the success of new tools be measured? What
  new lessons about visualization for communication are being revealed
  by the COVID-19 pandemic? For example, should physical distancing and
  economic disruption change communicative visualizations? How can data
  visualization help to fight against misleading facts and
  disinformation? What tools, platforms, and approaches have been useful
  to dispel untruths?
\item
  We particularly encourage contributors to address and illustrate
  issues like these with visual case studies that demonstrate the
  success or failure of communicative visualization projects in data
  journalism, public health and more. Our goal is to consider a broad
  range of examples and learn from their design decisions and process.
\item
  We invite contributions from any discipline, but particularly
  encourage journalists and designers to submit their work involving
  data-based communication or reporting. Scientific contributions
  concerning visualization for communication are of course welcome, as
  well.
\end{itemize}

Late-Breaking Works in Progress

\begin{itemize}
\item
  The purpose of this category is to present work in progress and
  receive feedback from attendees.
\item
  For research that is in progress, this session will provide a
  supportive atmosphere for helpful feedback and fresh perspectives on
  your aims and/or methods. Recommended structure for your one-page
  submission is: introduction, preliminary methods, preliminary findings
  (if applicable), and questions for attendees.
\item
  For practitioners, this is an opportunity to present contributions
  that showcase innovative visualizations or provide provocations for
  new ideas to emerge. Your one-page brief should include project
  background, design objectives, methods or design process, links to
  visualization design alternatives, preliminary findings (if
  available), and questions for attendees.
\end{itemize}

\hypertarget{content}{%
\section{Content}\label{content}}

Aim:

\begin{itemize}
\item
  visualizing uncertainty
\item
  important aspect of visualization for analysis. while often ignored in
  visualization for communication.
\item
  it's already challenging to express uncertainty in the context of
  prediction of future events, such as election polls or COVID-19
  prognoses. but what if there's uncertainty in something that we
  consider to be `facts'?
\item
  for example, in headline ``17\% of phd students experience burnout
  symptoms'', the percentage is an estimate. the real value may lay
  within a bound of uncertainty. that's because we infer something from
  a sample to a population. we can visualize this uncertainty with error
  bars on bar charts or shaded areas in line graphs.
\item
  but if we measure an entire population, such as the US census, there
  are no error bars in the figures. yet still, there may be uncertainty.
\item
  this type of uncertainty is the topic of this project: uncertainty due
  to missing data. missing data may occur across observations (e.g.,
  some people have no home address and cannot be reached) or within
  observations (e.g., some topics are sensitive and will not elicit
  responses). whatever the reason of the missingness, it may influence
  any {[}subsequent{]} estimates {[}down the line{]}.
\item
  what are intuitive ways to express uncertainty due to missing data to
  non-expert audiences?
\end{itemize}

Methods:

\begin{itemize}
\item
  missing data is ubiquitous and often ignored
\item
  but it can gravely influence estimates
\item
  gold standard in science is to `impute' (i.e., fill in) the missing
  entries before analyzing the data.
\item
  motivating example: relation between height and weight. quite trivial,
  but apparent that there should be some coherence.
\item
  developing an online evaluation suite to inspect missing data, impute
  it, and run analyses. We need visualizations for each step of this
  process. Visualizations for the first two steps exist, but not for the
  third.
\item
  \url{hanneoberman.shinyapps.io/shinymice-demo}
\end{itemize}

Question for attendees:

\begin{itemize}
\tightlist
\item
\end{itemize}

\hypertarget{todo}{%
\section{TODO}\label{todo}}

\begin{itemize}
\item
  look up mckinley (tableau -\textgreater{} gerkovink.com/slvrepo),
  tuftee (message of dataviz), wilkinson (grammar of graph)
\item
  add misleading facts/disinformation?
\item
  add link to public health/data journalism
\end{itemize}

\end{document}
